\documentclass[psfig,11pt]{article} 
 
\textheight 24.4cm
\textwidth 16.59cm
\topmargin -0.0cm
\oddsidemargin -0.04in
\evensidemargin -0.04in

\def\normalstretch{1.0}
\def\talllinestretch{1.5}
\def\shortlinestretch{0.8}
\renewcommand{\baselinestretch}{\normalstretch}
\parskip 1ex
\renewcommand{\topfraction}{.75}
\renewcommand{\textfraction}{.20}
\renewcommand{\floatsep}{0.0pt}
\renewcommand{\textfloatsep}{1.0cm}
\renewcommand{\floatpagefraction}{.85}

\def\portindent{\hspace{\parindent}}
\newcommand{\REM}[1]{} 
%\newcommand{\sol}[1]{\noindent\textcolor{red}{{#1}}}
%\newcommand{\solspace}[1]{}
\newcommand{\sol}[1]{\REM{#1}}
\newcommand{\solspace}[1]{\vspace{#1}}

\newif\ifsol
%\soltrue

\newcommand{\Bsol}{\ifsol \noindent \color{red}}
\newcommand{\Esol}[1]{\color{black} 
\else 
\vspace{{#1}} 
 \fi
 }

\input epsf
\usepackage{epsfig}
\usepackage{subfigure}
\usepackage{fullpage}
\usepackage{multirow}
%\usepackage{multicolumn}
\pagestyle{plain}
\usepackage[usenames]{color}
\usepackage{fancyhdr}

\lhead{CMPUT 229 - Fall 2014}
\chead{}
\rhead{\hspace{2in}{\Large Name: \ifsol \color{red} Solution \color{black} \fi  \hspace{3in}}}
\cfoot{}
\lfoot{}
\rfoot{}

\begin{document}
\ \\
\ \\
\ \\
\ \\
{\centering

 \thispagestyle{fancy}


\newcounter{quest}
\newenvironment{question}[1][??]{\noindent{\bf Question  \stepcounter{quest}
\arabic{quest} (\textrm{#1} points): }}

\renewcommand{\theenumi}{\alph{enumi}}

\begin{figure*}[t]
\begin{center}
\includegraphics[width=6.5in]{FindMax.pdf} 
\caption{\label{fig:FindMax} MIPS Assembly code for {\tt FindMax} procedure.}
\end{center}
\end{figure*}

\begin{question}[10]
The code for the {\tt FindMax} procedure shown in Figure~\ref{fig:FindMax} is not optimized and there are several improvements that could be made to improve this code.
\begin{enumerate}
\item ({\bf 6 points}) What changes would you do to the code to improve its performance. You do not need to write assembly code --- although you may if you wish. You can simply explain how you would transform the code to make it more efficient.

\ifsol
\color{red}
The optimization shown in Figure~\ref{fig:FindMax_o1} moves loop-independent code, such as the {\tt i*N} multiplication outside of the loops and transforms the loop to eliminate the execution of jump instructions inside the loop.

\begin{figure*}[t]
\begin{center}
\includegraphics[width=5.0in]{FindMax_Opt1.pdf} \caption{\label{fig:FindMax_o1} Optimization for FindMax, move loop-independent code outside of loop and use pointers instead of indices.}
\end{center}
\end{figure*}

The optimization shown in Figure~\ref{fig:FindMax_o2} recognizes that the rows of a two-dimensional array are stored in memory sequentially. Thus, the entire computation can be executed as a single loop that scans all the elements of the two-dimensional array.

\begin{figure*}[t]
\begin{center}
\includegraphics[width=5.0in]{FindMax_Opt2.pdf} 
\caption{\label{fig:FindMax_o2} Optimization 2 for {\tt FindMax}, recognize that two-dimensional array is stored as a single-dimensional array in memory.}
\end{center}
\end{figure*}

\color{black}
\else
\vspace{1in}
\fi

\item({\bf 4 points}) Given the same invocation of {\tt FindMax} with {\tt N = 10000} and {\tt M = 5000} that we studied in the previous question, would your changes increase or decrease the CPI? Explain your reasoning.

\ifsol
\color{red}

In both cases the CPI should increase because we reduce the number of instructions with low cycle count that are executed in the loop but we still have to execute the {\tt lw} in each iteration of the loop and this slower instruction dominates the CPI. For instance, for the code in Figure~\ref{fig:FindMax_o2}, the following holds:

\begin{eqnarray}
 \mathrm{Number\ of\ Instructions} &=& 10 + 4.5 \times N \times M \nonumber \\
 &=& 10+ 5 \times 10000 \times 5000 \nonumber \\
 &=& 225,000,010 \nonumber \\
\mathrm{Number\ of\ Cycles} &=& 6+2+10+(10+2.5+4) \times N \times M \nonumber \\
&=& 18+16.5 \times 10000 \times 5000 \nonumber \\
&=& 825,000,018 \nonumber \\
\mathrm{CPI} &=& 3.7 \frac{\mathrm{Clocks}}{\mathrm{Instructions}} \nonumber
\end{eqnarray}
\color{black}
\else
\vspace{1in}
\fi


\end{enumerate}

\end{question}

\end{document}

