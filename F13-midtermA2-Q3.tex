\documentclass[psfig,11pt]{article} 
 
\textheight 24.4cm
\textwidth 16.59cm
\topmargin -0.0cm
\oddsidemargin -0.04in
\evensidemargin -0.04in

\def\normalstretch{1.0}
\def\talllinestretch{1.5}
\def\shortlinestretch{0.8}
\renewcommand{\baselinestretch}{\normalstretch}
\parskip 1ex
\renewcommand{\topfraction}{.75}
\renewcommand{\textfraction}{.20}
\renewcommand{\floatsep}{0.0pt}
\renewcommand{\textfloatsep}{1.0cm}
\renewcommand{\floatpagefraction}{.85}

\def\portindent{\hspace{\parindent}}
\newcommand{\REM}[1]{} 
%\newcommand{\sol}[1]{\noindent\textcolor{red}{{#1}}}
%\newcommand{\solspace}[1]{}
\newcommand{\sol}[1]{\REM{#1}}
\newcommand{\solspace}[1]{\vspace{#1}}

\newif\ifsol
\soltrue

\newcommand{\Bsol}{\ifsol \noindent \color{red}}
\newcommand{\Esol}[1]{\color{black} 
\else 
\vspace{{#1}} 
 \fi
 }

\input epsf
\usepackage{epsfig}
\usepackage{subfigure}
\usepackage{fullpage}
\usepackage{multirow}
%\usepackage{multicolumn}
\pagestyle{plain}
\usepackage[usenames]{color}

\begin{document}

\newcounter{quest}
\newenvironment{question}[1][??]{\noindent{\bf Question  \stepcounter{quest}
\arabic{quest} (\textrm{#1} points): }}

\begin{question}[25]
The code for function {\tt make\_big\_endian} in the C programming language is as follows:

\begin{verbatim}
00  struct page_list {
01                  page_list*       prev;
02                  page_list*       next;
03                  unsigned int     page;
04                 };
05  int  *page_count;
06  char valid[1000]; 
07
08  void  make_big_endian(page_list **page_pointers)
09  {
10    page_list		*page_array;
11    unsigned int      i;
12    ...
13    valid[i-1] = valid[i];
14    *page_pointers++;
15    page_array++;
16    *page_pointers = page_array;
17    *page_count = i;
18     ....
19  }
\end{verbatim}

In this code {\tt page\_array} is a dinamically allocated array of {\tt page\_lists} (the actual allocation call is ommitted above) and {\tt page\_pointers} is an array of pointers to {\tt page\_list}. Assume that the variable {\tt page\_array} is stored in the stack at the address given by {\tt \$fp-4}; the global variable {\tt page\_count} is at the address given by {\tt \$gp} and the global array {\tt valid} starts at the address {\tt \$gp+4}. The parameter passed to {\tt make\_big\_endian} is the address of the first position of the array of pointers to {\tt page\_list} called {\tt page\_pointers}.  Assume that in this architecture an integer is stored in 32 bits and a memory address also occupies 32 bits. Assume that {\tt i} is in {\tt \$t0}. Write MIPS assembly code for each one of the following statements in the program above:

\begin{enumerate}
\item ({\bf 5 points})   {\tt valid[i-1] = valid[i];}
\ifsol
\color{red}
\begin{verbatim}
     # valid is an array of chars, thus no need to multiply index
     addi    $t2, $gp, $t0  #  $t2 <-- $gp+i = Address(valid[i-4])
     lb      $t3, 4($t2)    #  $t3 <-- MEM[$gp+i+4] = valid[i]
     sb      $t3, 3($t2)    #  MEM[$gp+i+3] = valid[i-1] <-- valid[i]
\end{verbatim}
\color{black}
\else

\renewcommand{\baselinestretch}{\talllinestretch}
%\begin{table}[h!] 
%\centering
\begin{tabular}{|ccc|} 
\hline
\ \ &\ \ \ \ \ \ \ \ \ \ \ \ \ \ \ \ \ \ \ \ \ \ \ \ \ \ \ \ \ \ \ \ \ \ \ \ \ \ \ \ \ \ \ \ \ \ \ \ \ \ \ \ \ \ \ \ \ \ \ \ \ \ \ \ \ \ \ \ \ \ \ \ \ \ \ \ \ \ \ \ \ \ \ \ \ \ \ \ \ \ \ \ \ \ \ \ \ \ \ \ \ \ \ \ \ \ \ \ \ \ \ \ \ &\ \ \\ \cline{2-2}
\ &\ &\ \\ \cline{2-2}
\ &\ &\ \\ \cline{2-2}
\ &\ &\ \\ \cline{2-2}
\ &\ &\ \\ \cline{2-2}
\ &\ &\ \\ \cline{2-2}
\ &\ &\ \\ \cline{2-2}
\ &\ &\ \\ \hline
\end{tabular}
%\end{table}

\fi

\newpage

\item ({\bf 5 points})	 {\tt *page\_pointers++;}
\ifsol
\color{red}
\begin{figure}[h!]
\centering
\includegraphics[scale=0.35]{figs/F13-midtermA2-Q3/Item2.pdf}
\caption{\label{fig:Item2} Illustration of {\tt *page\_pointers++;}}
\end{figure}

\begin{verbatim}
     # *page_pointers is a pointer to a page_list
     lw      $t2, 0($a0)     #  $t2 <-- *page_pointers
     addi    $t2, $t2, 12     #  $t2 <-- $t2+12
     sw      $t2, 0($a0)     #  *page_pointers <-- *page_pointers + 12
\end{verbatim}
\color{black}
\else

\renewcommand{\baselinestretch}{\talllinestretch}
%\begin{table}[h!] 
%\centering
\begin{tabular}{|ccc|} 
\hline
\ \ &\ \ \ \ \ \ \ \ \ \ \ \ \ \ \ \ \ \ \ \ \ \ \ \ \ \ \ \ \ \ \ \ \ \ \ \ \ \ \ \ \ \ \ \ \ \ \ \ \ \ \ \ \ \ \ \ \ \ \ \ \ \ \ \ \ \ \ \ \ \ \ \ \ \ \ \ \ \ \ \ \ \ \ \ \ \ \ \ \ \ \ \ \ \ \ \ \ \ \ \ \ \ \ \ \ \ \ \ \ \ \ \ \ &\ \ \\ \cline{2-2}
\ &\ &\ \\ \cline{2-2}
\ &\ &\ \\ \cline{2-2}
\ &\ &\ \\ \cline{2-2}
\ &\ &\ \\ \cline{2-2}
\ &\ &\ \\ \cline{2-2}
\ &\ &\ \\ \cline{2-2}
\ &\ &\ \\ \hline
\end{tabular}
%\end{table}
\fi


\item ({\bf 5 points})   {\tt page\_array++;}
\ifsol
\color{red}
\begin{verbatim}
     # page_array points to a page_list which is formed by 12 bytes
     lw      $t1, -4($fp)    #  $t1 <-- page_array
     addi    $t1, $t1, 12   #  $t1 <-- page_array+1
     sw      $t1, -4($fp)    #  page_array <- page_array+1
\end{verbatim}
\color{black}
\else

\renewcommand{\baselinestretch}{\talllinestretch}
%\begin{table}[h!] 
%\centering
\begin{tabular}{|ccc|} 
\hline
\ \ &\ \ \ \ \ \ \ \ \ \ \ \ \ \ \ \ \ \ \ \ \ \ \ \ \ \ \ \ \ \ \ \ \ \ \ \ \ \ \ \ \ \ \ \ \ \ \ \ \ \ \ \ \ \ \ \ \ \ \ \ \ \ \ \ \ \ \ \ \ \ \ \ \ \ \ \ \ \ \ \ \ \ \ \ \ \ \ \ \ \ \ \ \ \ \ \ \ \ \ \ \ \ \ \ \ \ \ \ \ \ \ \ \ &\ \ \\ \cline{2-2}
\ &\ &\ \\ \cline{2-2}
\ &\ &\ \\ \cline{2-2}
\ &\ &\ \\ \cline{2-2}
\ &\ &\ \\ \cline{2-2}
\ &\ &\ \\ \cline{2-2}
\ &\ &\ \\ \cline{2-2}
\ &\ &\ \\ \hline
\end{tabular}
%\end{table}
\fi

\item ({\bf 5 points})   {\tt *page\_pointers = page\_array;}
\ifsol
\color{red}
\begin{figure}[h!]
\centering
\includegraphics[scale=0.35]{figs/F13-midtermA2-Q3/Item4.pdf}
\caption{\label{fig:Item2} Illustration of {\tt *page\_pointers = page\_array;}
\end{figure}
\begin{verbatim}
     lw    $t2, -4($fp)     #  $t2 <-- page_array
     sw    $t2, 0($a0)    #  *page_pointers <-- page_array
\end{verbatim}
\color{black}
\else

\renewcommand{\baselinestretch}{\talllinestretch}
%\begin{table}[h!] 
%\centering
\begin{tabular}{|ccc|} 
\hline
\ \ &\ \ \ \ \ \ \ \ \ \ \ \ \ \ \ \ \ \ \ \ \ \ \ \ \ \ \ \ \ \ \ \ \ \ \ \ \ \ \ \ \ \ \ \ \ \ \ \ \ \ \ \ \ \ \ \ \ \ \ \ \ \ \ \ \ \ \ \ \ \ \ \ \ \ \ \ \ \ \ \ \ \ \ \ \ \ \ \ \ \ \ \ \ \ \ \ \ \ \ \ \ \ \ \ \ \ \ \ \ \ \ \ \ &\ \ \\ \cline{2-2}
\ &\ &\ \\ \cline{2-2}
\ &\ &\ \\ \cline{2-2}
\ &\ &\ \\ \cline{2-2}
\ &\ &\ \\ \cline{2-2}
\ &\ &\ \\ \cline{2-2}
\ &\ &\ \\ \cline{2-2}
\ &\ &\ \\ \hline
\end{tabular}
%\end{table}
\fi

\item ({\bf 5 points})   {\tt *page\_count = i;}
\ifsol
\color{red}
\begin{verbatim}
     lw     $t2, 0($gp)     #  $t2 <-- Address(page_count)
     sw     $t0, 0($t2)     #  *page_count <-- i
\end{verbatim}
\color{black}
\else

\renewcommand{\baselinestretch}{\talllinestretch}
%\begin{table}[h!] 
%\centering
\begin{tabular}{|ccc|} 
\hline
\ \ &\ \ \ \ \ \ \ \ \ \ \ \ \ \ \ \ \ \ \ \ \ \ \ \ \ \ \ \ \ \ \ \ \ \ \ \ \ \ \ \ \ \ \ \ \ \ \ \ \ \ \ \ \ \ \ \ \ \ \ \ \ \ \ \ \ \ \ \ \ \ \ \ \ \ \ \ \ \ \ \ \ \ \ \ \ \ \ \ \ \ \ \ \ \ \ \ \ \ \ \ \ \ \ \ \ \ \ \ \ \ \ \ \ &\ \ \\ \cline{2-2}
\ &\ &\ \\ \cline{2-2}
\ &\ &\ \\ \cline{2-2}
\ &\ &\ \\ \cline{2-2}
\ &\ &\ \\ \cline{2-2}
\ &\ &\ \\ \cline{2-2}
\ &\ &\ \\ \hline
\end{tabular}
%\end{table}
\fi

\end{enumerate}
\end{question}


\end{document}

