\documentclass[psfig,11pt]{article} 
\usepackage{graphicx} 
\usepackage{setspace}
\textheight 24.4cm
\textwidth 16.59cm
\topmargin -0.0cm
\oddsidemargin -0.04in
\evensidemargin -0.04in

\def\normalstretch{1.0}
\def\talllinestretch{1.5}
\def\shortlinestretch{0.8}
\renewcommand{\baselinestretch}{\normalstretch}
\parskip 1ex
\renewcommand{\topfraction}{.75}
\renewcommand{\textfraction}{.20}
\renewcommand{\floatsep}{0.0pt}
\renewcommand{\textfloatsep}{1.0cm}
\renewcommand{\floatpagefraction}{.85}

\def\portindent{\hspace{\parindent}}
\newcommand{\REM}[1]{} 
%\newcommand{\sol}[1]{\noindent\textcolor{red}{{#1}}}
%\newcommand{\solspace}[1]{}
\newcommand{\sol}[1]{\REM{#1}}
\newcommand{\solspace}[1]{\vspace{#1}}

\newif\ifsol
%\soltrue


\input epsf
\usepackage{epsfig}
\usepackage{subfigure}
\usepackage{fullpage}
\usepackage{multirow}
\pagestyle{plain}
\usepackage[usenames]{color}

\begin{document}

\newcounter{quest}
\newenvironment{question}[1][??]{\noindent{\bf Question  \stepcounter{quest}
\arabic{quest} (\textrm{#1} points): }}

\begin{question}[20] 
For each of the following statements you have three options: (i) leave
it blank; (ii) Mark it with {\bf T} to indicate that the statement is
true; (iii) Mark it with {\tt F} to indicate that the statement is
false.  \underline{\bf You lose 2 points for each incorrect answer.}
You win 4 points for each correct answer. You don't win or lose any
points if you leave the statement blank.  Regardless of how many
statements you mark wrong, your score in this question cannot be below
zero.

\begin{enumerate}

\item (\hspace{0.2in}\sol{\bf F\ }) In an embedded processor whose
  design was derived from the MIPS architecture, the address field for
  a branch instruction is 8 bits. The execution procedure for a branch
  is the same as in MIPS:
\begin{verbatim}
bne  $s3, $s4, 8:      PC <-- PC + 4
                       if($1 != $2) then PC <-- PC + (address << 2)
\end{verbatim}
The largest distance that a branch can jump backward in this architecture is 128 instructions.

\sol{The largest negative 8-bit address is 1000 0000 = - $2^7$ =
  - 128. The PC is incremented by 4 before this value is shifted and
  added to the PC, thus the largest jump backward with a branch is 127
  instructions.}

\item (\hspace{0.2in}\sol{\bf F\ }) Assume two cache designs $C_A$ and
  $C_B$ have the same block size. $C_A$ is a 16 KB 2-way set
  associative cache and $C_B$ is an 8 KB direct-mapped cache. More
  bits are used for index in $C_A$ than in $C_B$.

\sol{$C_A$ has twice as many blocks as $C_B$, but each index points to
  a set that contains two blocks. Thus both caches use the same number
  of bits for indexing.}

\item (\hspace{0.2in}\sol{\bf T\ }) The following code correctly
  execute an atomic swap between the value stored in the address
  specified by {\tt \$s1} and the content of register {\tt \$s4}:
\begin{verbatim}
try:          add   $t0, $zero, $s4
              ll    $t1, 0($s1)
              sc    $t0, 0($s1)
              beq   $t0, $zero, try
              add   $s4, $zero, $t1
\end{verbatim}

\item (\hspace{0.2in}\sol{\bf F\ }) To ensure the correct operation of
  the system, an exception handler must save the value that it uses
  into a special frame stored in the stack of the program that got
  interrupted by the exception.

\item (\hspace{0.2in}\sol{\bf T\ }) The figure below depicts the
  architecture of a 5-stage pipelined implementation of the data path
  for the MIPS architecture. This drawing is incorrect because the
  instruction fetched at time $T_i$ will write to the register
  specified by the instruction fetched at time $T_{i+3}$.
\begin{center}
       \includegraphics[width=6in]{Pipeline}
\end{center}

\end{enumerate}
\end{question}

\end{document}

