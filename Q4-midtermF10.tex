\documentclass[psfig,11pt]{article} 
 
\textheight 24.4cm
\textwidth 16.59cm
\topmargin -0.0cm
\oddsidemargin -0.04in
\evensidemargin -0.04in

\def\normalstretch{1.0}
\def\talllinestretch{1.5}
\def\shortlinestretch{0.8}
\renewcommand{\baselinestretch}{\normalstretch}
\parskip 1ex
\renewcommand{\topfraction}{.75}
\renewcommand{\textfraction}{.20}
\renewcommand{\floatsep}{0.0pt}
\renewcommand{\textfloatsep}{1.0cm}
\renewcommand{\floatpagefraction}{.85}

\def\portindent{\hspace{\parindent}}
\newcommand{\REM}[1]{} 
%\newcommand{\sol}[1]{\noindent\textcolor{red}{{#1}}}
%\newcommand{\solspace}[1]{}
\newcommand{\sol}[1]{\REM{#1}}
\newcommand{\solspace}[1]{\vspace{#1}}


\newif\ifsol
%\soltrue

\input epsf
\usepackage{epsfig}
\usepackage{subfigure}
\usepackage{fullpage}
\usepackage{multirow}
%\usepackage{multicolumn}
\pagestyle{plain}
\usepackage[usenames]{color}

\begin{document}

\newcounter{quest}
\newenvironment{question}[1][??]{\noindent{\bf Question  \stepcounter{quest}
\arabic{quest} (\textrm{#1} points): }}

\renewcommand{\theenumi}{\alph{enumi}}

\color{black}

\begin{question}[25] In the reverse engineering of a computer application, 
an important task is to write the source code for a corresponding
segment of assembly code. You are working for a security agency and
you have been asked to provide source-level code for the function {\tt
  enigma} whose assembly code is below. The code is printed
with line numbers to facilitate referencing to instructions in your
answer.


\begin{verbatim}
(1) enigma:
(2)       add  $t0, $zero, $zero
(3)       slt  $t1, $t0, $a2
(4)       beq  $$t1, $zero, label_one
(4') label_two:
(5)       sll  $t2, $t0, 2
(6)       add  $t3, $a1, $t2
(7)       lw   $t4, 0($t3)
(8)       sll  $t5, $t4, 2
(9)       add  $t6, $a0, $t5
(10)      lw   $t7, 0($t6)
(11)      lw   $t8, 4($t3)
(12)      sll  $t9, $t8, 2
(13)      add  $t1, $a0, $t9
(14)      sw   $t7, 0($t1)
(15)      addi $t0, $t0, 1
(15')     blt  $t0, $a2, label_two
(16) label_one:
(17)      jr $ra
\end{verbatim}

\begin{enumerate}

\item ({\bf 10 points}) Assume that this function follows the MIPS
  procedure-calling conventions. How many parameters does the function
  {\tt enigma} has? Give a name for each parameter. You you will use
  these names in your source code. Also, indicate the type of the
  parameter (is it an address or a value? In the case of a value, can
  you say anything about the number of bits?) Justify your answer.


\ifsol
\color{red}
The function accesses registers {\tt \$a0}, {\tt \$a1}, and {\tt \$a2}. Therefore it has three parameters. 
\\
{\tt \$a2} is compared with register {\tt \$t0} which was initialized
to zero. Thus it must be a value. It could have any number of bits
from 1 to 32. We will call it {\tt N}.
\\
{\tt \$a1} is added to {\tt \$t2} at line 6, and the result ({\tt
  \$t3}) is then used as the base for a load. Thus {\tt \$a1} is
probably the base of an array, and thus is a pointer. We will call it
{\tt C}
\\
{\tt \$a0} is added to {\tt \$t5} in line 9, and the result ({\tt
  \$t6}) is used as the base for a load. Thus {\tt \$a1} is also
probably a memory address. We will call it {\tt A} 
\color{black}
\else
\vspace{1in}
\fi

\item ({\bf 10 points}) How many memory loads and how many memory stores are
  executed by {\tt enigma}? Your answer can be in terms of one or more
  of the parameters of the function.

\ifsol
\color{red}
 There is a loop in {\tt enigma} and each iteration of the loop
  executes 3 loads and 1 store. The loop executes {\tt \$a2} times (we
  called it {\tt N} above). Thus 3 $\times$ {\tt N} loads and {\tt
    N} stores are executed.
\color{black}
\else
\vspace{1in}
\fi

\item ({\bf 5 points}) Write C-style source code that leads to the
  generation of the assembly code above for {\tt enigma}.


\ifsol
\color{red}
The first step is to write comments in the assembly code. We have
  given the following names to the parameters: {\tt \$a0 = A, \$a1 =
    C, \$a2 = N}. It is easy to see that {\tt \$t0} is the index for a loop,
  thus we will call it {\tt i}. 

\begin{verbatim}
(1) enigma:
(2)       add  $t0, $zero, $zero       # i <-- 0
(3)       slt  $t1, $t0, $a2           # if (i < N) then $t1 <-- 1 else $t1 <-- 0
(4)       beq  $$t1, $zero, label_one  # if ($t1 == 0) goto label_one
(4') label_two:
(5)       sll  $t2, $t0, 2             # $t2 = 4*i
(6)       add  $t3, $a1, $t2           # $t3 = C+4*i
(7)       lw   $t4, 0($t3)             # $t4 = Mem[C+4*i]
(8)       sll  $t5, $t4, 2             # $t5 = 4 * Mem[C+4*i]
(9)       add  $t6, $a0, $t5           # $t6 = A + (4 * Mem[C+4*i])
(10)      lw   $t7, 0($t6)             # $t7 = Mem[A + (4 * Mem[C+4*i])]
(11)      lw   $t8, 4($t3)             # $t8 = Mem[C+4*i + 4]
(12)      sll  $t9, $t8, 2             # $t9 = Mem[C+4*i + 4] * 4
(13)      add  $t1, $a0, $t9           # $t1 = A + Mem[C+4*i + 4] * 4
(14)      sw   $t7, 0($t1)             # Mem[A + Mem[C+4*i + 4] * 4] 
                                       #    <-- Mem[A + (4 * Mem[C+4*i])]
(15)      addi $t0, $t0, 1             # i <-- i+1
(15')     blt  $t0, $a2, label_two     # if(i < N) goto label_two
(16) label_one:
(17)      jr $ra
\end{verbatim}

One of the first things that we can do to simplify this code is to
  recognize that a reference to the memory position {\tt Mem[C+4*i]}
  is represented in the C language as {\tt C[i]}. Similarly, a reference to {\tt Mem[C+4*i + 4]} is written in the C language as {\tt C[i+1]}. With these
  replacements, we can rewrite the comments above as follows:
  
\begin{verbatim}
(1) enigma:
(2)       add  $t0, $zero, $zero       # i <-- 0
(3)       slt  $t1, $t0, $a2           # if (i < N) then $t1 <-- 1 else $t1 <-- 0
(4)       beq  $$t1, $zero, label_one  # if ($t1 == 0) goto label_one
(4') label_two:
(5)       sll  $t2, $t0, 2             # $t2 = 4*i
(6)       add  $t3, $a1, $t2           # $t3 = C+4*i
(7)       lw   $t4, 0($t3)             # $t4 = C[i]
(8)       sll  $t5, $t4, 2             # $t5 = 4 * C[i]
(9)       add  $t6, $a0, $t5           # $t6 = A + 4*C[i]
(10)      lw   $t7, 0($t6)             # $t7 = Mem[A + (4*C[i])]
(11)      lw   $t8, 4($t3)             # $t8 = C[i+1]
(12)      sll  $t9, $t8, 2             # $t9 = C[i+1] * 4
(13)      add  $t1, $a0, $t9           # $t1 = A + C[i+1] * 4
(14)      sw   $t7, 0($t1)             # Mem[A + C[i+1] * 4] 
                                       #    <-- Mem[A + (4*C[i])]
(15)      addi $t0, $t0, 1             # i <-- i+1
(15')     blt  $t0, $a2, label_two     # if(i < N) goto label_two
(16) label_one:
(17)      jr $ra
\end{verbatim}

Now we should apply the same transformations to the reminder
  memory references to obtain the following pseudo code:

\begin{verbatim}
(1) enigma:
(2)       add  $t0, $zero, $zero       # i <-- 0
(3)       slt  $t1, $t0, $a2           # if (i < N) then $t1 <-- 1 else $t1 <-- 0
(4)       beq  $$t1, $zero, label_one  # if ($t1 == 0) goto label_one
(4') label_two:

          A[C[i+1]] <-- A[C[i]]

(15)      addi $t0, $t0, 1             # i <-- i+1
(15')     blt  $t0, $a2, label_two     # if(i < N) goto label_two
(16) label_one:
(17)      jr $ra
\end{verbatim}

Thus the C code for {\tt enigma} is
  (versions of the code that use more temporary variables for storage
  of intermediate results, or use slightly different types, are also
  acceptable):

\begin{verbatim}
void enigma(int *A, int *C, int N)
{
 int i;

 for(i=0 ; i < N ; i++)
    A[C[i+1]] <-- A[C[i]]
}    
\end{verbatim}
\color{black}

\fi

\end{enumerate}

\end{question}



\end{document}

