\documentclass[psfig,11pt]{article} 
 
\textheight 24.4cm
\textwidth 16.59cm
\topmargin -3.0cm
\oddsidemargin -0.04in
\evensidemargin -0.04in

\def\normalstretch{1.0}
\def\talllinestretch{1.8}
\def\shortlinestretch{0.8}
\renewcommand{\baselinestretch}{\normalstretch}
\parskip 1ex
\renewcommand{\topfraction}{.75}
\renewcommand{\textfraction}{.20}
\renewcommand{\floatsep}{0.0pt}
\renewcommand{\textfloatsep}{1.0cm}
\renewcommand{\floatpagefraction}{.85}

\def\portindent{\hspace{\parindent}}
\usepackage[usenames]{color}
\usepackage{fancyhdr}

\lhead{CMPUT 229 - Final - Winter 2014}
\chead{}
%\rhead{\hspace{2in}{\Large Name: \ifsol \color{red} Solution \color{black} \fi  \hspace{3in}}}
\cfoot{}
\lfoot{}
\rfoot{}
\pagestyle{plain}
\newcommand{\REM}[1]{} 
\newcommand{\sol}[1]{\noindent\textcolor{red}{{#1}}}
\newcommand{\solspace}[1]{}
%\newcommand{\sol}[1]{\REM{#1}}
%\newcommand{\solspace}[1]{\vspace{#1}}
\newif\ifsol
 \soltrue

\input epsf
\usepackage{epsfig}
\usepackage{subfigure}
\usepackage{fullpage}
\thispagestyle{fancy}

\begin{document}

\renewcommand{\baselinestretch}{\talllinestretch}  
\ \\
\ \\
%\noindent {\center \bf \Large CMPUT 229 - Quiz \# 2 - Fall 2014}\\
%\framebox{\Large Name: \ifsol \color{red} Solution \color{black} \fi  \hspace{4in}\hrulefill}
\renewcommand{\baselinestretch}{\normalstretch}

\newcounter{quest}
 \stepcounter{quest}
\newenvironment{question}[1][??]{\noindent{\bf Question  \stepcounter{quest}
\arabic{quest} (\textrm{#1} points): }}

\renewcommand{\theenumi}{\alph{enumi}}


\begin{question}[30]

  The 16-bit {\it half precision} floating point representation has the following specification:


\renewcommand{\baselinestretch}{\talllinestretch}
\begin{table}[!h]\centering
\begin{tabular}{|c|lcr|lcr|}
\multicolumn{1}{l}{15} &
\multicolumn{1}{l}{14} &  & \multicolumn{1}{r}{10} & 
\multicolumn{1}{l}{9} &  & \multicolumn{1}{r}{0} \\ \hline
\multicolumn{1}{|c|}{\ \ S\ \ } &
\multicolumn{3}{|c|}{\ \ \ \ \ \ \ $\mathit{biasedexponent}$ \ \ \ \ \ \ \ \ \ \ \ \ \ }  &
\multicolumn{3}{|c|}{\ \ \ \ \ \ \ \ \ \ \ \ \ \ \ \ \ $\mathit{fraction}$ \ \ \ \ \ \ \ \ \ \ \ \ \ \ \ \ \ \ \ \ \ \ \ \ }  \\ \hline
\end{tabular}
\end{table}

\begin{eqnarray}
N &=& \left\{ \begin{array}{ll}
(-1)^S \times 0.0 & \mbox{if $\mathit{biasedexponent} = 0$ and $\mathit{fraction} = 0$}\\  
(-1)^S \times 0.\mathit{fraction} \times 2^{-14} & \mbox{if $\mathit{biasedexponent} = 0$ and $\mathit{fraction} \neq 0$}\\
(-1)^S \times 1.\mathit{fraction} \times 2^{\mathit biasedexponent - 15} & \mbox{if $0 < \mathit{biasedexponent} < 31$}\\
(-1)^S \times \infty & \mbox{if $\mathit{biasedexponent} = 31$ and $\mathit{fraction} = 0$}\\
NaN                  & \mbox{if $\mathit{biasedexponent} = 31$ and $\mathit{fraction} \neq 0$}
              \end{array} 
\right.       \nonumber
\end{eqnarray}

\begin{enumerate}
\item ({\bf 4 points}) what is the binary representation of -37.375 in the half-precision floating-point representation?



\begin{table}[!h]\centering
\begin{tabular}{|c|lcr|lcr|}
\multicolumn{1}{l}{15} &
\multicolumn{1}{l}{14} &  & \multicolumn{1}{r}{10} & 
\multicolumn{1}{l}{9} &  & \multicolumn{1}{r}{0} \\ \hline
\multicolumn{1}{|c|}{\ \ \sol{1} \ \ } &
\multicolumn{3}{|c|}{\ \ \ \ \ \ \ \sol{10100} \ \ \ \ \ \ \ \ \ \ \ \ \ }  &
\multicolumn{3}{|c|}{\ \ \ \ \ \ \ \ \ \ \ \ \ \ \ \ \ \sol{0010 1011 00} \ \ \ \ \ \ \ \ \ \ \ \ \ \ \ \ \ \ \ \ \ \ \ \ }  \\ \hline
\end{tabular}
\end{table}
\color{black}


\ifsol
\color{red}
\vspace{.2in}
$-37.375_{10} =  -100101.011_2 = 1.00101011 \times 2^5$ \\
$\mathit{biasedexponent} - 15 = 5 \Rightarrow \mathit{biasedexponent} = 20 = 01010_2$\\
$\mathit{sign} = 1$ \\
$\mathit{fraction} = 0010101100$ \\


 \color{black}
 \else
 \vspace{.1in}
 \fi
 
 
\item ({\bf 6 points})  {\tt A = 0x6404} and {B = 0x4790} are two half-precision floating pointing. What is the true value of {\tt A + B}  expressed in decimal notation? That is, if we have infinite precision to do the addition and store the result, what is {\tt A + B}?

\item ({\bf 8 points}) What is the value of {\tt A} and the value of {\tt B}? Express each of these values both in normalized base-two notation and in decimal notation.

\begin{table}[!h]\centering
\begin{tabular}{l|c|lcr|lcr|}
\multicolumn{1}{l}{}& \multicolumn{1}{l}{15} &
\multicolumn{1}{l}{14} &  & \multicolumn{1}{r}{10} & 
\multicolumn{1}{l}{9} &  & \multicolumn{1}{r}{0} \\ \cline{2-8}
{\tt A =} & \multicolumn{1}{|c|}{\ \ \sol{0}\ \ } &
\multicolumn{3}{|c|}{\ \ \ \ \ \ \ \sol{11110} \ \ \ \ \ \ \ \ \ \ \ \ \ }  &
\multicolumn{3}{|c|}{\ \ \ \ \ \ \ \ \ \ \ \ \ \ \ \ \ \sol{00 0000 0000} \ \ \ \ \ \ \ \ \ \ \ \ \ \ \ \ \ \ \ \ \ \ \ \ }  \\ \cline{2-8}
\end{tabular}
\end{table}

 \ifsol
\color{red}
$A = (-1)^0 \times 1.0 \times 2^{30-15} = 1.0 \times 2^{15}$\\
$A = 1000\ 0000\ 0000\ 0000_2 = 2^{10} \times 2^5 = 1024 \times 32 = 32768_{10}$\\
\color{black}
\else
\vspace{0.7in}
\fi

\begin{table}[!h]\centering
\begin{tabular}{l|c|lcr|lcr|}
\multicolumn{1}{l}{}& \multicolumn{1}{l}{15} &
\multicolumn{1}{l}{14} &  & \multicolumn{1}{r}{10} & 
\multicolumn{1}{l}{9} &  & \multicolumn{1}{r}{0} \\ \cline{2-8}
{\tt B =} & \multicolumn{1}{|c|}{\ \ \sol{0}\ \ } &
\multicolumn{3}{|c|}{\ \ \ \ \ \ \ \sol{10011} \ \ \ \ \ \ \ \ \ \ \ \ \ }  &
\multicolumn{3}{|c|}{\ \ \ \ \ \ \ \ \ \ \ \ \ \ \ \ \ \sol{01 0000 0000} \ \ \ \ \ \ \ \ \ \ \ \ \ \ \ \ \ \ \ \ \ \ \ \ }  \\ \cline{2-8}
\end{tabular}
\end{table}

\ifsol
\color{red}
$B = (-1)^0 \times 1.01 \times 2^{19-15} = 1.01 \times 2^{4}$\\
$B = 10100_2 = 16+4 = 20_{10}$\\
\color{black}
\else
 \vspace{0.7in} 
\fi


\newpage
\item{\bf 4 points}) What is the true value of $A+B$ expressed in decimal notation? In other words, what is the value of $A+B$ if an infinite precision could be used to compute the addition and to store the result?

\ifsol 
\color{red}
$A + B = 32768+20 = 32788_{10}$\\
\color{black}
\else
 \vspace{0.7in} 
\fi

\item ({\bf 5 points}) Assume a floating-point unit uses the NVIDIA format presented above. This unit has no guard, no round, and no sticky bits. What is the value of $A+B$, expressed both in normalized base-two notation and in decimal notation, computed by this machine?

\ifsol
\color{red}
To align {\tt A} with {\tt B}, we need to move the binary point of {\tt B} eleven positions to the left. Therefore:

 $ B =  0.0000\ 0000\ 0010\ 1\times 2^{15}$
 
 
  \begin{verbatim}
          mantissa      
  A = + 1.0000 0000 00
  B = + 0.0000 0000 00
----------------------------------------------
A+B =   1.0000 0000 00
\end{verbatim}

Therefore $A+B = B = 1.0 \times 2^{15} = 32768_{10}$
\color{black}
\else
\vspace{2.5in} % Solution Space
\fi



\item ({\bf 5 points}) Assume a floating-point unit uses the NVIDIA format presented above. This unit has one guard, one round, and one sticky bit. What is the value of $A+B$, expressed in normalized base-two notation, computed by this machine?

\ifsol
\color{red}
 \begin{verbatim}
          mantissa       Guard  Round Sticky 
  A = + 1.0000 0000 00|    0     0       0
  B = + 0.0000 0000 00|    1     0       1
----------------------------------------------
A+B =   1.0000 0000 00|    1     0       1
\end{verbatim}

Now we have to round up because of the sticky bit. Therefore the result is:

$A+B = 1.0000\ 0000\ 01 \times 2^{15} = 1000\ 0000\ 0010\ 0000_2 =  32768 + 32 = 32800_{10}$

\color{black}
\else
\vspace{0.5in}
\fi
\end{enumerate}
\end{question}





\end{document}
