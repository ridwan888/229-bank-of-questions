\documentclass[psfig,11pt]{article} 
\usepackage{graphicx} 
\usepackage{setspace}
\textheight 24.4cm
\textwidth 16.59cm
\topmargin -0.0cm
\oddsidemargin -0.04in
\evensidemargin -0.04in

\def\normalstretch{1.0}
\def\talllinestretch{1.5}
\def\shortlinestretch{0.8}
\renewcommand{\baselinestretch}{\normalstretch}
\parskip 1ex
\renewcommand{\topfraction}{.75}
\renewcommand{\textfraction}{.20}
\renewcommand{\floatsep}{0.0pt}
\renewcommand{\textfloatsep}{1.0cm}
\renewcommand{\floatpagefraction}{.85}

\def\portindent{\hspace{\parindent}}
\newcommand{\REM}[1]{} 
%\newcommand{\sol}[1]{\noindent\textcolor{red}{{#1}}}
%\newcommand{\solspace}[1]{}
\newcommand{\sol}[1]{\REM{#1}}
\newcommand{\solspace}[1]{\vspace{#1}}
\newif\ifsol
%\soltrue

\input epsf
\usepackage{epsfig}
\usepackage{subfigure}
\usepackage{fullpage}
\usepackage{multirow}
\pagestyle{plain}
\usepackage[usenames]{color}
\renewcommand{\theenumi}{\alph{enumi}}
\newcounter{quest}
\newenvironment{question}[1][??]{\noindent{\bf Question  \stepcounter{quest}
\arabic{quest} (\textrm{#1} points): }}

\begin{document}

\color{black}

\begin{question}[20]
In this question you will demonstrate that you understand how the data stored in caches is organized. Assume that each cache operates in a machine with a 32-bit virtual address.


\begin{itemize}
\item({\bf 10 points}) For each of the cache configurations below, indicate how many bits are used for byte offset, for cache index, and for tag. 

\doublespacing
\begin{tabular}{|c|c|c|c|c|c|} \hline
Cache capacity & Associativity       & Size of Cache Block & Tag Size & Index Size & Byte offset size \\ \hline
64 KB                   & Direct mapped  & 128 bytes                   & \sol{16} & \sol{9} & \sol{7} \\ \hline
32 KB                   & 2-way                  & 64 bytes                     & \sol{18}   & \sol{8} & \sol{6} \\ \hline
32 MB                   & 8-way                 & 512 bytes                   & \sol{10} & \sol{13} & \sol{9} \\ \hline
\end{tabular}
\singlespacing


\ifsol
\color{red}
\begin{eqnarray}
\mathrm{\#\ of\ blocks\ in\ cache} &=& \frac{\mathrm{cache\ capacity}}{\mathrm{block\ size}} \nonumber \\
\mathrm{\#\ of\ entries\ in\ cache} &=& \frac{\mathrm{\#\ of\ blocks\ in\ cache}}{\mathrm{associativity}} \nonumber \\
\mathrm{\#\ of\ index\ bits} &=& log_2 (\mathrm{\#\ of\ entries\ in\ cache}) \nonumber \\
\mathrm{\#\ of\ offset\ bits} &=& log_2(\mathrm{size\ of\ cache\ block}) \nonumber \\
\mathrm{\# of\ tag\ bits} &=& 32 -  \mathrm{\#\ of\ index\ bits} - \mathrm{\#\ of\ offset\ bits} \nonumber
\end{eqnarray}

For first cache:
\begin{eqnarray}
\mathrm{\#\ of\ blocks\ in\ cache} &=& \frac{64 KB}{128} = \frac{2^{16}}{2^7} = 2^9 \nonumber \\
\mathrm{\#\ of\ entries\ in\ cache} &=& \frac{2^9}{1} = 2^9 \nonumber \\
\mathrm{\#\ of\ index\ bits} &=& log_2 (2^9) = 9 \mathrm{\ bits} \nonumber \\
\mathrm{\#\ of\ offset\ bits} &=& log_2(128) = 7 \mathrm{\ bits} \nonumber \\
\mathrm{\# of\ tag\ bits} &=& 32 - 9 -7 = 16 \mathrm{\ bits} \nonumber
\end{eqnarray}

For second cache:
\begin{eqnarray}
\mathrm{\#\ of\ blocks\ in\ cache} &=& \frac{32 KB}{64} = \frac{2^{15}}{2^6} = 2^9 \nonumber \\
\mathrm{\#\ of\ entries\ in\ cache} &=& \frac{2^9}{2} = 2^8 \nonumber \\
\mathrm{\#\ of\ index\ bits} &=& log_2 (2^8) = 8 \mathrm{\ bits} \nonumber \\
\mathrm{\#\ of\ offset\ bits} &=& log_2(64) = 6 \mathrm{\ bits} \nonumber \\
\mathrm{\# of\ tag\ bits} &=& 32 - 8 -6 = 18 \mathrm{\ bits}  \nonumber
\end{eqnarray}

For third cache:
\begin{eqnarray}
\mathrm{\#\ of\ blocks\ in\ cache} &=& \frac{32 MB}{512} = \frac{2^{25}}{2^9} = 2^{16} \nonumber \\
\mathrm{\#\ of\ entries\ in\ cache} &=& \frac{2^{16}}{8} = 2^{13} \nonumber \\
\mathrm{\#\ of\ index\ bits} &=& log_2 (2^{13}) = 13 \mathrm{\ bits} \nonumber \\
\mathrm{\#\ of\ offset\ bits} &=& log_2(512) = 9 \mathrm{\ bits} \nonumber \\
\mathrm{\# of\ tag\ bits} &=& 32 - 13 - 9 = 10 \mathrm{\ bits}  \nonumber
\end{eqnarray}

\color{black}
\else
\vspace{2.0in}
\fi

\item{\bf 10 points} The table below contains the same cache organizations listed in the table above. Using binary notation, provide the value of the tag and the value of the cache index for a memory reference to the address 0xABCD EF78. Also indicate which word within the block is being accessed. The word at offset 0 is word \#0, the word at offset 4 is word \#1, etc.
\end{itemize}

\doublespacing
\begin{tabular}{|c|c|c|c|c|c|} \hline
Cache & Associativity       & Cache Block & \ \ \ \ \ \ \ \ \ \ \ \ \ Tag \ \ \ \ \ \ \ \ \ \ \ \ \ & \ \ \ \ \ \ \ \ \ \ Index \ \ \ \ \ \ \ \ \ & word accessed \\
capacity & & & & & \\ \hline
64 KB                   & Dir. map.  & 128 bytes                   & \sol{1010 1011 1100 1101} & \sol{1110 1111 0}&  \sol{30}\\
32 KB                   & 2-way                  & 64 bytes                     & \sol{1010 1011 1100 1101 11}& \sol{1011 1101} & \sol{14} \\
32 MB                   & 8-way                 & 512 bytes                   & \sol{1010 1011 11} & \sol{00 1101 1110 111} & \sol{94} \\ \hline
\end{tabular}
\singlespacing


\ifsol
\color{red}
64KB, Direct, 128 bytes:

\begin{tabular}{ccccccccc}
                                    A       &  B  &       C  &      D  &       E  &      F  &      \multicolumn{2}{c}{7}  &        8 \\
                                 1010 & 1011 & 1100 & 1101 & 1110 & 1111&  0& 111&  1000\\
                                 \multicolumn{4}{|c|}{tag} &  \multicolumn{3}{|c|}{index} & \multicolumn{2}{|c|}{offset}\\
\end{tabular}

\vspace{0.2in}
32KB, 2-way, 64 bytes:

\begin{tabular}{cccccccccc}
                                    A       &  B  &       C  &      D  &       \multicolumn{2}{c}{E}  &      F  &    \multicolumn{2}{c}{7}  &        8 \\
                                 1010 & 1011 & 1100 & 1101 & 11&10 & 1111&  01&11&  1000\\
                                 \multicolumn{5}{|c|}{tag} &  \multicolumn{3}{|c|}{index} & \multicolumn{2}{|c|}{offset}\\
\end{tabular}

\vspace{0.2in}
32MB, 8-way, 512 bytes:


\begin{tabular}{cccccccccc}
                                    A       &  B  &       \multicolumn{2}{c}{C}  &      D  &       E  &      \multicolumn{2}{c}{F}  &    7  &        8 \\
                                 1010 & 1011 & 11& 00 & 1101 & 1110 & 111&1&  0111&  1000\\
                                 \multicolumn{3}{|c|}{tag} &  \multicolumn{4}{|c|}{index} & \multicolumn{3}{|c|}{offset}\\
\end{tabular}
\color{black}
\fi

\end{question}


\end{document}

