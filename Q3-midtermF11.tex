\documentclass[psfig,11pt]{article} 
 
\textheight 24.4cm
\textwidth 16.59cm
\topmargin -0.0cm
\oddsidemargin -0.04in
\evensidemargin -0.04in

\def\normalstretch{1.0}
\def\talllinestretch{1.5}
\def\shortlinestretch{0.8}
\renewcommand{\baselinestretch}{\normalstretch}
\parskip 1ex
\renewcommand{\topfraction}{.75}
\renewcommand{\textfraction}{.20}
\renewcommand{\floatsep}{0.0pt}
\renewcommand{\textfloatsep}{1.0cm}
\renewcommand{\floatpagefraction}{.85}

\def\portindent{\hspace{\parindent}}
\newcommand{\REM}[1]{} 
%\newcommand{\sol}[1]{\noindent\textcolor{red}{{#1}}}
\newcommand{\sol}[1]{\REM{#1}}

\newif\ifsol
%\soltrue

\input epsf
\usepackage{epsfig}
\usepackage{subfigure}
\usepackage{fullpage}
\usepackage{multirow}
%\usepackage{multicolumn}
\pagestyle{plain}
\usepackage[usenames]{color}

\begin{document}

\newcounter{quest}
\newenvironment{question}[1][??]{\noindent{\bf Question  \stepcounter{quest}
\arabic{quest} (\textrm{#1} points): }}


\color{black}

\begin{question}[25] A common pattern of execution in numerical computing is called a {\em stencil} computation. In a stencil computation the same computing pattern is repeated for multiple elements of an array. An example is a computation that computes the average of the surrounding elements in an $N\times N$ array. Assume that an integer is stored using 4 bytes in this machine. Also assume that the dimension of the vector {\tt A} was declared statically to be $64\times 64$

\begin{verbatim}
00 int stencil2D_element_update(int *A, int i, int j)
01 {
02   int sum;
03   
04   sum = A[i][j-1];
05   sum = sum + A[i][j+1];
06   sum = sum + A[i-1][j];
07   sum = sum + A[i+1][j];
08   return(sum/4.0);
09 }
\end{verbatim}

Write MIPS assembly code for the computation of {\tt  sum} (lines 04 to 08 in the C code)\\ \underline{using a minimum number of instructions}.

\ifsol
 \color{red}
 \begin{tabular}{lll}
\verb| sll| & \verb| $t0, $a1, 8   |     & \verb| # $t0 <-- 4*i*N = 4*i*64 = 256*i = i<<8 |\\
\verb| sll| & \verb| $t1, $a2, 2   |     & \verb| # $t1 <-- 4*j |\\
\verb| add| & \verb| $t2, $a0, $t0|     & \verb| # $t2 <-- A + $t0 = A + 4*i*N |\\
\verb| add| & \verb| $t2, $t2, $t1 |     & \verb| # $t2 <-- A + 4*i*N + j*4 = Address(A[i][j]) |\\
\verb| lw| & \verb| $t3, -4($t2)  |     & \verb| # sum = $t3 <-- A[i][j-1] |\\
\verb| lw| & \verb| $t4, 4($t2)   |     & \verb| # $t4 <- A[i][j+1] |\\
\verb| add| & \verb| $t3, $t3, $t4|      & \verb| # sum <-- sum + A[i][j+1] |\\
\verb| lw| & \verb| $t4, -256($t2)|   & \verb| # $t4 <- A[i-1][j] |\\
\verb| add| & \verb| $t3, $t3, $t4  |    & \verb| # sum <- sum + A[i-1][j] |\\
\verb| lw| & \verb| $t4, 256($t2)|    & \verb| # $t4 <- A[i+1][j] |\\
\verb| add | & \verb| $t3, $t3, $t4  |    & \verb| # sum <- sum + A[i+1][j] |
\end{tabular}
 \color{black}
 \else
 \vspace{0.5in}
 \fi
\end{question}

\end{document}

